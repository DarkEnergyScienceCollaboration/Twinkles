% ====================================================================
\chapter*{Preamble}
\label{chp:preamble}
\addcontentsline{toc}{chapter}{Preamble}
% ====================================================================

This document describes the Twinkles tiny simulated sky survey
project. In this preamble we  provide a high level sketch of the
project, including the motivation and philosophy behind it, and a
overview of the survey design.

% --------------------------------------------------------------------

\section{Philosophy}
\label{sec:preamble:philosophy}

Both the type Ia supernovae (SN) and time delay strong lenses (SL)
cosmological probes rely on accurate measurements of object
variability given multi-filter light curve data, for large numbers of
rare objects (relative to galaxies). Our ability to do cosmography
using these objects depends on our ability to make these measurements
{\it accurately} given the LSST data. Both the SN and SL working
groups need to develop software to find these objects, extract high
fidelity light curves for them, and infer model parameters from those
light curves. Understanding the systematic errors introduced at each
step in this analysis requires us to work on {\it realistic mock
data,} where we can quantify both the residual bias and the additional
random uncertainty resulting from our measurement software.

The Twinkles approach is to define a tiny patch of sky, and simulate
an over-abundance of supernovae and strong lenses in it, while making
everything else (observing strategy, astrophysical foregrounds, image
generation, and LSST DM stack processing) as realistic as possible.
The idea is to produce a single dataset that allows tests of both SL
and SN light curve measurement accuracy, using the current version of
the  DM analysis software -- and then develop the DESC analysis
software  against this test dataset.

Both the DM and DESC software will evolve over
the period leading up to first light, suggesting a staged approach to
generating and analyzing the mock data. The Twinkles survey is planned
to span the first two DESC data challenge eras, with the DC2 dataset (\TwinklesTwo) being
more complex (both to generate and to analyze) than the \TwinklesOne
dataset.

As well as building the analysis software itself, the DESC needs to
develop infrastructure to carry out the simulation and analysis
computations: the LSST DM simulation and processing software must be
tracked, and run at the scale required by the data challenges.
Twinkles is a pathfinder for this DESC infrastructure, driving its
development over the first two data challenge eras.

% --------------------------------------------------------------------

\section{Design}
\label{sec:preamble:design}

\autoref{tab:preamble:design} summarizes the high level differences
between the two stages of the Twinkles survey.
The design of each stage is likely to change with time, in response to
developments in the capabilities of both the LSST DM simulation and
image processing software.



\begin{table*}[!h]
\begin{center}
\caption{Twinkles 1 and 2: High Level Design Comparison.}
\label{tab:preamble:design}
\small
\begin{tabularx}{0.9\linewidth}{*{3}{>{\centering\arraybackslash}X}}
  \hline
  \hline          & \TwinklesOne & \TwinklesTwo \\
  \hline
  Survey area
                  & \multicolumn{2}{c}{\centering $\sim 100$~arcmin$^2$ within a Deep Drilling Field}
                      \\
  Observing Strategies
                  & \multicolumn{2}{c}{\centering WFD, DDF}
                       \\ \hline
  Supernovae
                  & Type Ia
                     & Multiple types; Realistic positions
                       \\
  Strong Lenses
                  & Quasars  
                     & Quasars, SNe; Hosts.
                       \\
  Filter set
                  & $ugrizy^a$
                     & $ugrizy^a$
                    \\
  Dithers$^b$
               & Small: $\sim 10$~arcsec
                  & Large: as for LSS
                    \\
  Sensors used
               & Central only
                  & As dithered to
                    \\
  Realism
               & Intermediate
                  & High
                    \\
  Campaign Length
               & $3$~seasons
                  & $10$~seasons
                    \\
  \hline
  \multicolumn{3}{p{0.8\linewidth}}{\scriptsize Notes: a) A fallback position
  for \TwinklesOne is for it to be single-filter -- this was the baseline
  vision of the DESC Science Roadmap.
  b) Field rotation will be included in both \TwinklesOne and \TwinklesTwo.}
\end{tabularx}
\normalsize
\medskip\\
\end{center}
\end{table*}

% --------------------------------------------------------------------

\section{This Document}
\label{sec:preamble:meta}

This purpose of this document is to provide a repository for the
current working specifications of the Twinkles datasets, and  the
consequent demands they place on the software needed to generate and
analyze them. It is divided into two chapters, one for \TwinklesOne
and one for \TwinklesTwo.  Apart from some introductory sections
written to supply scientific context, the sections of each chapter
are written as very terse lists of requirements, in a form that can be
easily included in the LSST DESC DC1 Requirements document.

% ====================================================================
