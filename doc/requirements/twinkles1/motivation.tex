% ====================================================================
\section{Motivation}
\def\secname{\chpname:motivation}
\label{\secname}
% ====================================================================

\contact{Phil Marshall}{@drphilmarshall},
\contact{Michael Wood-Vasey}{@wmwv},
\contact{Richard Dubois}{@richardxdubois}

Type Ia supernovae and time delay lenses share the property that their
light curves must be extracted with sufficient fidelity that time
domain model parameters (such as stretch and time delay) can be
inferred. Our ability to do cosmography using these objects depends on
our ability to make these measurements {\it accurately} given the LSST
data. In the next section we define a set of analyses that we need to do
as part of this accuracy demonstration program, and that can be done
in the 2016--2017 time frame using available simulation and analysis
technology and computing infrastructure.

A note on terminology: we refer to the process of light curve
extraction as ``Monitoring.'' Supernova light curve extraction will be
performed by a  tool referred to as \SNMonitor, and strong lens light
curve extraction  by a related piece of software called \SLMonitor.
For strong lenses, the key parameter to be inferred  from a set of
light curves is the time delay, and so we refer to the software tool
that  performs that inference as \SLTimer. Supernova light curve
parameters  are typically inferred with tools known as
``light curve fitters,'' to be deployed by the code \SNDistance.

% A note here on what we are *not* trying to investigate?

% A note here on infrastructure build-up?

% ====================================================================
