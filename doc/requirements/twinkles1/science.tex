% ====================================================================
\section{Science Analysis}
\def\secname{\chpname:science}
\label{\secname}
% ====================================================================

\contact{Phil Marshall}{@drphilmarshall},
\contact{Michael Wood-Vasey}{@wmwv},

Below we describe the supernova and strong lensing science analysis
that we want to do with the \TwinklesOne data. In each case we first
introducing the measurement issues we face, and then define the
investigations of them that we want to do. These then dictate the
requirements we have on the challenge dataset design.

% --------------------------------------------------------------------

\subsection{Supernovae}
\label{\secname:supernovae}

Introduction to supernova analysis in \TwinklesOne.

% - - - - - - - - - - - - - - - - - - - - - - - - - - - - - - - - - -

\subsubsection{Light Curve Extraction Issues}
\label{\secname:supernovae:monitor}

Photometric calibration.

Forced photometry accuracy.

Host galaxy light contamination.


% - - - - - - - - - - - - - - - - - - - - - - - - - - - - - - - - - -

\subsubsection{Distance Measurement Issues}
\label{\secname:supernovae:distance}

Host galaxy photometry and structure.

Photometric error accuracy.

% - - - - - - - - - - - - - - - - - - - - - - - - - - - - - - - - - -

\subsubsection{Proposed Analyses}
\label{\secname:supernovae:analyses}


% --------------------------------------------------------------------

\subsection{Strong Lensing}
\label{\secname:stronglensing}

From the first Time Delay Challenge we know that around 400 lensed
quasars (defined here as the ``Gold Sample'') should be measurable
with cosmological accuracy with LSST, provided that a) 6 day cadence
can be achieved and b) the light curves extracted from the LSST images
are as clean as those in the challenge.

6-day cadence requires 5 filters to be used: TDC2 will test
assumption a) above, that  a 5-6 filter light curve can be modeled as
accurately as a single filter light curve. The fidelity of the light
curves depends on our ability to extract them, and this requires image
simulations of very high realism to be analyzed with the tools of
sufficient sharpness.

In \TwinklesOne we will test assumption b), and assess the fidelity of
lensed quasar light curves as observed and measured with the LSST system.

% - - - - - - - - - - - - - - - - - - - - - - - - - - - - - - - - - -

\subsubsection{Light Curve Extraction Issues}
\label{\secname:stronglensing:monitor}

Point image separation (deblending).

Photometric accuracy (forced photometry).

Lens and host galaxy light contamination.


% - - - - - - - - - - - - - - - - - - - - - - - - - - - - - - - - - -

\subsubsection{Time Delay Measurement Issues}
\label{\secname:stronglensing:timedelay}

Correlated photometric error accuracy and mitigation.


% - - - - - - - - - - - - - - - - - - - - - - - - - - - - - - - - - -

\subsubsection{Proposed Analyses}
\label{\secname:stronglensing:analyses}

We propose to answer the following questions:
\begin{itemize}
\item Was the quality of the DC1 Gold Sample photometry realistic?
\item How can we better model LSST lensed quasar photometry in future time delay challenges?
\end{itemize}

For this we will need the \TwinklesOne survey to have the following
properties:
\begin{itemize}
\item The field should contain a significant random fraction (at least
25\%,  and preferably 100\%) of the TDC1 Gold Sample of 400 lensed
quasars,  which should vary in the same way as the TDC1 objects (at
least with regard to their  AGN variability, which domiates over
microlensing).
\item The survey should simulate 10 years of LSST observing in
wide-deep-fast strategy, with realistic observing conditions.
\item Images should be in either $r$ or $i$-band, as assumed in TDC1.
\item The survey can be single filter, but the mean night to night
cadence needs to be 6 days, to allow comparison with TDC1.
\end{itemize}

After light curve extraction we will then perform the following tests:
\begin{itemize}
\item The noise properties of the \TwinklesOne and TDC1 light curves
will be summarized and compared.
\item Time delays will be measured for each system using the fiducial
TDC1 algorithm, and the mean accuracy compared against that in TDC1.
\end{itemize}


% ====================================================================
